\chapter{EPSI SPROUL- November 2017}
This purpose of this experiment is to gather EPSI data during a two-day local San Diego cruise.  EPSI will be mounted on a new FCTD-style MMP replacement.  The intention is to create a simple dropping vehicle for the FCTD boom/winch.  

\section{Cruise}
\begin{itemize}
    \item Ship: RV Sproul 
    \item People:  Sean Lastuka, Arnaud Le Boyer, Sara Goheen, Michael Goldin, Matthew H. Alford, Jonathan Ladner
    \item Destination:  La Jolla Canyon, depth minimum 200m.  
    \item Mob:  Nov 8-9, 2017
    \item Set Sail:  0700, Nov 8, 2017
    \item Demob:  0700, Nov 9, 2017
    \item Lat/Long of test area:  32.588°N 117.436°W, Depth = 675m, about 12 miles WSW of Point Loma. 
\end{itemize}

We will test the FCTD with the 'tubed' Epsi. We will also have an MMP instrument and winch.  We will test basic functionality of both systems including data comms and internal recording on the deck during mob.  We will also confirm the Epsilometer is successfully gathering data. When we arrive on station, we do a half-dozen MMP deployments (with and without the Epsi strapped to the side) to a depth of 100-300m.  We will then move to the new tubed Epsi system deployed off the MMP. Profiles will continue on station for the duration of the voyage.  At least three people are required for MMP profiles, so we will want to work in shifts to maximize the available time.  We will then move over to our. It is not expected the epsilometer will require a battery change during the cruise.  The MMP may need a battery charge during the exercise. The MMP and FCTD controller will be using 110V.  The FCTD will need ships 440V 3-phase power. We will have the spare MMP winch on deck in case of FCTD system failure.  We will not need the ship's CTD.


\subsection{EPSI configuration}
Hardware, firmaware
ADC config, Probe serial number, Sv, madre rev, map rev, Anti aliasing filters, charge amp TF, sinc 4, software


\section{Data}

